\newpage
\thispagestyle{empty}

\begin{par}
\setlength{\baselineskip}{5mm}

\begin{center}
	\textbf{АННОТАЦИЯ} \\
	\vspace{5mm}
	ВЫПУСКНАЯ КВАЛИФИКАЦИОННАЯ РАБОТА БАКАЛАВРА \\
	Наименование темы \uline{\topic\hfill} \\
	\hrulefill
\end{center}

\begin{flushleft}
	Выполнена студентом(кой) \uline{\studentforabstract\hfill} \\
	\faculty, \university \\
	Кафедра \uline{\department\hfill} \\
	Группа \uline{\group} \\
	Направление подготовки: \programnum\thinspace\programname \\
	Направленность (профиль): \specialization \\
    \vspace{5mm}
	% Объем работы на единицу меньше реального, так как аннотация с работой не переплетается
	Объем работы: \total{page} страниц \\
	Количество иллюстраций: \totalfigures \\
	Количество таблиц: \totaltables \\
	Количество литературных источников: \total{citenum} \\
	Количество приложений: \total{appendixnum} \\
	Ключевые слова: latex, шаблон, ещекакоенибудьключевоеслово
\end{flushleft}

Аннотация выпускной квалификационной работы должна содержать: название работы, сведения об объеме (количестве страниц), количестве иллюстраций и таблиц, количестве использованных источников, количестве приложений, перечень ключевых слов; текст аннотации(содержит формулировку задачи и основных результатов, их новизну и актуальность). 

Ключевые слова в совокупности дают представление о содержании. Ключевыми словами являются слова или словосочетания из текста работы, которые несут существенную смысловую нагрузку с точки зрения информационного поиска. Перечень включает от 5 до 15 ключевых слов (словосочетаний) в именительном падеже, напечатанных в строку через запятые прописными буквами. 

Текст аннотации должен отражать:
    \begin{enumerate}
    	\item объект исследования;
    	\item цель работы и ее актуальность;
        \item задачи, выполненные для достижения цели; 
    	\item метод исследования;
    	\item полученные результаты и их новизну;
    	\item область применения и рекомендации.
    \end{enumerate}

Излагать содержание аннотации необходимо в связанной повествовательной форме, но допускается и схематичное составление. Объем аннотации не должен превышать 1 страницы.

\begin{flushleft}
    \vspace{1.02cm}
	\vfill
	\student, \\ % Место для подписи
	\today
\end{flushleft}

\end{par}