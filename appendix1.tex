\chapter*{Приложение A}
\addcontentsline{toc}{chapter}{Приложение A}
\stepcounter{appendixnum}

\begin{center}
	Заголовок приложения
\end{center}

В приложения рекомендуется включать материалы, которые по каким-либо причинам не могут быть включены в основную часть:
\begin{enumerate}
	\item промежуточные математические доказательства, формулы и расчеты;
	\item таблицы вспомогательных цифровых данных;
	\item протоколы испытаний;
	\item описание аппаратуры, приборов, применяемых при проведении экспериментов;
	\item инструкции, методики, разработанные в процессе выполнения работы;
	\item иллюстрации вспомогательного характера;
	\item материалы, дополняющие выпускную квалификационную работу;
	\item копии собранных документов и т.п.
\end{enumerate}

% Сомнительно
\textbf{(Лучше уточнить у научного руководителя)} В случае, если в рамках ВКР разрабатывается программное решение или его компоненты, в Приложение в обязательном порядке включаются руководство пользователя и/или разработчика, а также описание программы, выполненные в соответствии с действующими стандартами на программные документы.
