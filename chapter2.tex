\chapter{Вторая глава с примерами картинок и таблиц}

Опечатки, обнаруженные в процессе подготовки работы, допускается исправлять подчисткой или закрашиванием белой краской (с использованием корректора) и нанесением на том же месте исправленного текста машинописным способом или черными чернилами, пастой или тушью --- рукописным способом.

\begin{figure}[htbp]
	\centering
	\includegraphics[width=0.33\textwidth]{sample.jpg}
	\caption{Пример картинки}
	\label{fig:sample-figure}
\end{figure}

Повреждения листов текста, помарки и следы неполностью удаленного прежнего текста не допускаются.

\section*{Единственный раздел второй главы (не нумеруется)}
\addcontentsline{toc}{section}{Единственный раздел второй главы}

Разделы, подразделы, пункты и подпункты следует нумеровать арабскими цифрами и записывать с абзацного отступа. Разделы должны иметь порядковую нумерацию в пределах всего текста
работы, за исключением приложений. Номер подраздела включает номер раздела и порядковый номер подраздела, разделенные точкой. Аналогично строятся номера пунктов и подпунктов.

Если раздел или подраздел имеет только один пункт или пункт имеет один подпункт, то \textit{их не нумеруют}.

% Таблицы удобно генерировать тут: tablesgenerator.com
\begin{longtable}[c]{|l|l|l|ll}
	\cline{1-3}
	\multicolumn{1}{|c|}{Что-то} &
	\multicolumn{1}{c|}{Еще что-то} &
	\multicolumn{1}{c|}{И еще что-то} &
	&
	\\ \cline{1-3}
	\endhead
	%
	& 123          & 321          &  &  \\ \cline{2-3}
	\multirow{-2}{*}{Объединенные ячейки} & 321          & 123          &  &  \\ \cline{1-3}
	\textbf{Еще строка}                   & \textbf{123} & \textbf{321} &  &  \\ \cline{1-3}
	\caption{Пример таблицы}
	\label{tab:table-sample}\\
\end{longtable}