\chapter{Вторая глава с примерами картинок и таблиц}

Иллюстрации (чертежи, графики, схемы, компьютерные распечатки, диаграммы, фотоснимки, карты) располагаются в работе непосредственно после текста, где они упоминаются впервые, или на следующей странице. Иллюстрации могут быть в компьютерном исполнении, в том числе и цветные.

\begin{figure}[htbp]
	\centering
	\includegraphics[width=0.33\textwidth]{sample.jpg}
	\caption{Пример картинки}
	\label{fig:figure-sample}
\end{figure}

На все иллюстрации должны быть даны ссылки в работе. Нумерация их сквозная (за исключением иллюстраций приложений) по всей работе арабскими цифрами. Допускается именовать иллюстрации в пределах раздела. В этом случае номер иллюстрации состоит из номера раздела и порядкового номера иллюстрации, разделенных точкой. Иллюстрация обозначается словом «рисунок». Порядковый номер рисунка и его наименование проставляются под рисунком посередине строки через тире (слово «Рисунок», его номер и через тире наименование). Точка после названия иллюстрации не ставится. Иллюстрации могут иметь пояснительные данные (подрисуночный текст), которые располагаются под рисунком. Еще ниже идет его обозначение с соответствующим номером и наименованием.

\section*{Единственный раздел второй главы (не нумеруется)}
\addcontentsline{toc}{section}{Единственный раздел второй главы}

Таблицы применяют для лучшей наглядности и удобства сравнения показателей. Таблица должна иметь название и порядковую нумерацию арабскими цифрами, сквозную по всей работе. Допускается нумеровать таблицы в пределах раздела. В этом случае номер таблицы состоит из номера раздела и порядкового номера таблицы, разделенных точкой. Название таблицы должно отражать ее содержание, быть точным и кратким. Оно помещается над таблицей слева без абзацного отступа в одну строку с ее номером через тире.

% Таблицы удобно генерировать тут: tablesgenerator.com
\begin{longtable}{|l|l|l|ll}
	\caption{Пример таблицы}
	\label{tab:table-sample}\\
	\cline{1-3}
	\multicolumn{1}{|c|}{Что-то} & \multicolumn{1}{c|}{Еще что-то} & \multicolumn{1}{c|}{И еще что-то} &  &  \\ \cline{1-3}
	\endhead
	%
	\multirow{2}{*}{Объединенные ячейки} & 123          & 321          &  &  \\ \cline{2-3}
	& 321          & 123          &  &  \\ \cline{1-3}
	\textbf{Еще строка}                  & \textbf{123} & \textbf{321} &  &  \\ \cline{1-3}
\end{longtable}