\chapter*{Заключение}
\addcontentsline{toc}{chapter}{Заключение}

Заключение содержит итоговые выводы теоретического и практического характера, к которым автор пришел в ходе исследования. В заключении должна содержаться оценка полноты решений поставленных задач, разработка рекомендаций и исходных данных по конкретному использованию результатов работы, в том числе аспекты внедрения результатов работы, дана оценка технико-экономической эффективности внедрения. Также следует указать пути и цели дальнейшей работы или обосновать нецелесообразность её продолжения.

\vfill

Выпускная квалификационная работа выполнена мной самостоятельно и с соблюдением правил профессиональной этики. Все использованные в работе материалы и заимствованные принципиальные положения (концепции) из опубликованной научной литературы и других источников имеют ссылки на них. Я несу ответственность за приведенные данные и сделанные выводы.

Я ознакомлен с программой государственной итоговой аттестации, согласно которой обнаружение плагиата, фальсификации данных и ложного цитирования является основанием для не допуска к защите выпускной квалификационной работы и выставления оценки «неудовлетворительно».

\begin{flushleft}
	\begin{minipage}[t]{0.35\textwidth}
		$\underset{\text{ФИО студента}}{\text{\uline{\student}}}$ \\[5mm]
		<<\uline{\hspace{9mm}}>>\uline{\hspace{33mm}}20\uline{\hspace{5mm}}г.
	\end{minipage}
	\hspace{35mm}
	\begin{minipage}[t]{0.35\textwidth}
		$\underset{\text{Подпись студента}}{\text{\uline{\hspace{0.55\textwidth}}}}$
	\end{minipage}
\end{flushleft}
