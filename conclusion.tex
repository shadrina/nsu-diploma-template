\chapter*{Заключение}
\addcontentsline{toc}{chapter}{Заключение}

Заключение содержит итоговые выводы теоретического и практического характера, к которым автор пришел в ходе исследования. В заключении должна содержаться оценка полноты решений поставленных задач, разработка рекомендаций и исходных данных по конкретному использованию результатов работы, в том числе аспекты внедрения результатов работы, дана оценка технико-экономической эффективности внедрения. Также следует указать пути и цели дальнейшей работы или обосновать нецелесообразность её продолжения.

Выпускная квалификационная работа выполнена мной самостоятельно и с соблюдением правил профессиональной этики. Все использованные в работе материалы и заимствованные принципиальные положения (концепции) из опубликованной научной литературы и других источников имеют ссылки на них. Я несу ответственность за приведенные данные и сделанные выводы. 

Я ознакомлен с программой государственной итоговой аттестации, согласно которой обнаружение плагиата, фальсификации данных и ложного цитирования является основанием для не допуска к защите выпускной квалификационной работы и выставления оценки «неудовлетворительно». 

\begin{center}
	\begin{minipage}[b]{0.35\textwidth}
		\studentfiotemplate
	\end{minipage}
	\hspace{25mm}
	\begin{minipage}[b]{0.35\textwidth}
        \studentsigntemplate
	\end{minipage}
\end{center}

\begin{minipage}{0.35\textwidth}
        \vspace{1.02cm}
        \studentdatetemplate
\end{minipage}

