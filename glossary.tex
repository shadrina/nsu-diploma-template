\chapter*{Определения, обозначения и сокращения}
\addcontentsline{toc}{chapter}{Определения, обозначения и сокращения}

\newcommand{\customglossaryentry}[2]
{
	\textit{#1} --- #2
}

\customglossaryentry{Определение}{объяснение (формулировка), раскрывающее, разъясняющее содержание, смысл чего-нибудь.}

\customglossaryentry{Обозначение}{знак, которым что-нибудь обозначено.}

\customglossaryentry{Сокращение}{сокращённое обозначение чего-нибудь.}

% Определения, обозначения и сокращения – это перечень условных обозначений, символов, принятых в работе сокращений, терминов. Если в работе принята специфическая терминология, а также употребляются мало распространенные сокращения, новые символы, обозначения и т.п., то их перечни должны быть представлены в работе в виде отдельных списков. Перечень должен располагаться столбцом, в котором слева приводят, например, сокращения, справа – его детальную расшифровку. Если в работе специальные термины, сокращения, символы, обозначения и т.п. повторяются не более трех раз, перечень не составляют, а их расшифровку приводят в тексте при первом упоминании. Запись определений, обозначений и сокращений идет в порядке упоминания в тексте работы с необходимой расшифровкой и пояснениями.