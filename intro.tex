\chapter*{Введение}
\addcontentsline{toc}{chapter}{Введение}

Во введении дается характеристика и обоснование выбора темы выпускной квалификационной работы, обосновывается актуальность проблемы, к которой относится тема работы, объект и предмет исследования, определяется цель и задачи, методы исследования. Кроме того, дается краткий обзор современного состояния данной проблемы в мире – степень разработанности темы, определена теоретическая база исследования, т.е. перечислены все наиболее значимые авторы, в том числе зарубежные, проводившие научные или научно-практические исследования по данной проблеме и должно быть сформулировано и обосновано отношение студента-выпускника к этим научным позициям (критический анализ изученной литературы (отечественной и зарубежной) и заключение по этому анализу). Далее следует постановка задачи и основной полученный результат в общих словах, научная новизна и практическая значимость работы. Обзор литературы не должен превышать 1/3 текста.
